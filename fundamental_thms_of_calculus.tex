\section{Teoremi fondamentali del calcolo integrale}
\subsubsection{(recall) Teorema fondamentale del calcolo integrale (Riemann)}
\circled{1}
Se $f\in C([a,b])$, e se definiamo una funzione $F(x)$ come l'integrale definito di $f$ da $a$ a $x$, ovvero:
$$F(x)=\int_a^x f(t)\mathrm dt$$
allora $F(x)\in C^1((a,b))$ e la sua derivata è uguale a $f(x)$:
$$F'(x)=f(x)$$
\circled{2}
Inoltre, se $F$ è una primitiva di $f$ su $[a,b]$, allora
$$\int_a^b f(x)\ \mathrm dx=F(b)-F(a)$$\label{(FFC)}
Ci chiediamo ora cosa succede se $f$ è soltanto $L^1$
\subsection{(thm) 1° Teorema fondamentale del calcolo integrale (*)}
Presi $f\in L^1([a,b]),\ F(x)=\int_a^xf(t)\ \mathrm dt$ allora:
\begin{enumerate}
    \item $F$ è differenziabile in a.e. $x\in [a,b]$
    \item $F'(x)=f(x)$ per a.e. $x\in [a,b]$
\end{enumerate}
\subsubsection{(def) Punto di Lebesgue di $f$}
$f\in \mathcal L^1([a,b]),\ x \in [a,b]$ è un punto di Lebesgue per $f$ se
$$\lim_{h\to 0}\frac 1h \int_x^{x+h}|f(t)-f(x)|\ \mathrm dt=0$$
(se $x=a$, allora $h\to 0^+$, se $x=b$, allora $h\to 0^-$)
\subsubsection{(thm) Lebesgue}
$f\in \mathcal L^1([a,b])\implies $ a.e. $x\in [a,b]$ is a Lebesgue point.
\subsection{Continuity of functions}
\subsubsection{(def) Absolutely Continuous (AC) functions}
Una funzione assolutamente continua ha tutte le proprietà di una funzione continua, ma con qualcosa in più: se scegliamo un gruppo di piccoli intervalli su cui la funzione è definita, la somma dei cambiamenti della funzione su ciascuno di questi intervalli sarà piccola, a patto che la somma delle lunghezze degli intervalli sia piccola.

In altre parole, per una funzione assolutamente continua, possiamo controllare quanto cambia il suo valore “globalmente” spezzando l’intervallo in tante piccole parti: se le parti sono piccole abbastanza, anche i cambiamenti della funzione saranno piccoli.
\paragraph{Formal definition}
$f:I\to \mathbb R$ è una funzione assolutamente continua $f\in AC(I)$ se $\forall \varepsilon>0,\ \exists \delta$ t.c. $\forall n\in \mathbb N$, $\forall$ families of disjoint subintervals of $I$, $\lambda\Big (\bigcup_{i=1}^n(a_i,b_i)\Big )<\delta$
$$\implies \sum_{i=1}^n|f(b_i)-f(a_i)|<\varepsilon$$
\paragraph{Author's remarks}
\begin{itemize}
    \item La composizione di due funzioni assolutamente continue è ancora assolutamente continua, a condizione che una delle funzioni sia limitata.
    \begin{proof}\ 
    
    Take
    \begin{itemize}
        \item $f:[a,b]\to \mathbb R\in AC([a,b])$
        \item $g:\mathbb R\to \mathbb R$, assolutamente continua su ogni intervallo chiuso e limitato
    \end{itemize}
    
    
    \end{proof}
\end{itemize}
\subsubsection{(def) Uniformly Continuous (UC) functions}
$f$ è uniformemente continua se $\forall \varepsilon>0\quad \exists \delta $ t.c., $\forall a_1,b_1\in I$,
$$|a_1-b_1|<\delta \implies |f(a_1)-f(b_1)|<\varepsilon$$
($\delta$ è indipendente da $a_1,b_1$)
$$UC(I)\supset AC(I)$$
\subsubsection{(def) Lipschitz Continuous functions}
If $\exists L>0$ t.c. $\forall x,y\in I$, $|f(x)-f(y)|\leq L|x-y|$
\paragraph{remarks}
\begin{itemize}
    \item The set of Lipschits-continuous functions is strictly contained in the set of AC functions.$$Lip(I)\subset AC(I)$$    
    \item La composizione di funzioni lipschitziane è ancora una funzione lipschitziana.
    \begin{proof}
    Consideriamo due funzioni $f,g\in Lip(I)$,\\
    Per definizione di funzione Lipschitz-continua,
    $\forall\ x,y\in I$:
    \begin{enumerate}
        \item $|f(x)-f(y)|\leq L_1|x-y|$
        \item $|g(x)-g(y)|\leq L_2|x-y|$
    \end{enumerate}
    Di conseguenza,
    $$|f(g(x))-f(g(y)|\leq L_1|g(x)-g(y)|\leq L_1L_2|x-y|$$
    $$L=L_1\cdot L_2$$
    $$\implies f\circ g=f(g(t))\in Lip(I)$$
    
    \end{proof}
\end{itemize}

\subsubsection{(rmk) anticipazione}
Vedremo che
$$Lip(I)\subsetneq  AC(I)\subsetneq  UC(I)$$
Vedremo che:
\begin{itemize}
    \item $g'\in C\iff g\in C^1$
    \item $g'\in L^1\iff g\in AC$
\end{itemize}
\subsection{(thm) 2° Teorema fondamentale del calcolo integrale (*)}
Sia $g:[a,b]\to \mathbb R$. The following propositions are equivalent:
\begin{enumerate}[label=\roman*]
    \item $g$ è assolutamente continua in $[a,b]$ $$g\in AC([a,b])$$
    \item 
    \begin{itemize}
        \item $g$ è differenziabile a.e. in $[a,b]$
        \item $g'\in L^1([a,b])$
        \item $g(x)-g(y)=\int_y^xg'(t)\mathrm dt\quad \forall x,y\in[a,b]$
    \end{itemize}    
\end{enumerate}
\paragraph{Corollary}
$$f\in L^1([a,b])\implies F\in AC([a,b])$$
\subsubsection{(thm) Absolute continuity of the integral (*)}
Let $f\in L^1(X,\mathcal M,\mu)$. Allora $\forall \varepsilon>0\quad \exists \delta>0$ tale che:
$$\begin{cases}
    E\in \mathcal M\\\mu(E)<\delta
\end{cases}\implies \int_E |f| \mathrm d \mu<\varepsilon$$
\subsubsection{(examples) AC not implies Lip continuity}
\textbf{Esempio 1}
Consider $f(x)=\sqrt x $ in $[0,1]$
$$\sqrt x = \int _0^x \frac 1{2\sqrt t }\mathrm dt\quad x\in [0,1]$$
e $g\in AC([0,1])$ (Indeed, $g'\in L^1(0,1)$) 
but $\sqrt{x}\notin $\ Lip\\
\subsubsection{(example) UC does not imply AC}
Considera $$g(x)=\begin{cases}x \sin\frac 1x\quad 0<x\leq 1\\0\quad\quad\quad\quad x=0\end{cases}$$
è continua in $[0,1]\implies g\in UC([0,1])$
But it is not AC.\\
Indeed,
$$ g'(x)=\sin \Big(\frac 1x\Big)-\frac 1x \cos\Big(\frac 1x\Big)\quad 0<x\leq 1$$
\begin{itemize}
    \item $\sin (\frac 1x)\in L^1((0,1))$\\ (si provi applicando il Dominate Convergence Theorem con g=1)
    \item $-\frac 1x \cos(\frac 1x) \notin L^1((0,1))$
\end{itemize}

\subsection{AC functions and weak derivatives}
Lavoriamo in $X=[a,b]\subset \mathbb R$ (diventa molto diverso in $\mathbb R^n$).
\subsubsection{(recall) Compactly supported ($\varphi \in C_0^\infty([a,b])$)}
\begin{itemize}
    \item $\varphi \in C^\infty$
    \item $\exists [c,d]\subset (a,b)$ t.c. $\varphi=0 $ in $ (a,b)\setminus [c,d]$
\end{itemize}
\subsubsection{(prop) intergration by parts in AC}
Take $u\in [a,b]\to \mathbb R$\\
$$u\in AC([a,b])\\\iff$$
\begin{itemize}
    \item $u \in C([a,b])$
    \item u è differenziabile a.e.
    \item $u'\in L^1([a,b])$
    \item $$\int_a^b u'\varphi \mathrm d x = -\int_a^b u\varphi'\mathrm d x\quad \forall \varphi \in C_0^\infty([a,b])$$
\end{itemize}

\subsubsection{(def) $W^{1,1}(a,b)$ Weak derivative}
Let $u\in L^1(a,b)$. \\
Diciamo che $$u \in W^{1,1}(a,b) \iff \exists w\in L^1(a,b) \text{ t.c. } \int_a^b u\varphi '\mathrm dx=-\int_a^b w\varphi \mathrm d x \quad \forall \varphi \in C_0^\infty $$

Such a $w$ is called weak derivative of $u$ in $(a,b)$, denoted $u'$
\subsubsection{(rmk) Remarks on weak derivatives}
\paragraph{rmk 0)}
\begin{itemize}
    \item $C([a,b])\iff L^1(a,b)$
    \item $C^1([a,b])\iff W^{1,1}(a,b)$
\end{itemize}
\paragraph{rmk 1)}
Both $u$ and $w=u'$ are equivalence classes
\paragraph{rmk 2)}
If such a $w$ exists, it is unique.
$$\int_a^b u \varphi ' = -\int_a^b w_1\varphi $$
$$\int_a^b u \varphi ' = -\int_a^b w_2\varphi $$
$\forall \varphi \in C_0^\infty ([a,b])$
$$\implies\int_a^b(w_1-w_2)\varphi =0\quad \forall \varphi \in C_0^\infty$$
$$\implies w_1-w_2=0\quad a.e. \text{ in} \ [a,b]$$
\paragraph{rmk 3)}
In principle, the pointwise and weak derivatives are different objects, and the notation $u'$ may be misleading.\\
But we know that if we take absolute continuous functions, they coincide.
\paragraph{rmk 4)}
The definition of weak derivative can be extended in the senses of measures, of distributions.

Take $$H(x)=\begin{cases} 0\quad x<0\\ 1\quad x>0\end{cases}$$
$$-\int_{-1}^1 H(x)\varphi '(x)\mathrm d x = -\int_0^1 \varphi'(x)\mathrm d x = -\varphi(1)+\varphi(0)$$
$$\varphi(1)=0$$

$$\varphi(0)=\int_{[-1,1]} \varphi \mathrm\ d\delta_0$$
This suggests that $H^1$ è pari a 0 almost everywhere (pointwise), ma è pari a $\delta_0$ (dirac) weakly
\subsubsection{(thm) AC and $W^{1,1}$ relation (*)}
$$u\in  AC([a,b])\iff u \in W^{1,1}([a,b])$$
