\section{Dual Spaces}

Consider $X$ normed space, $Y=\mathbb R$
$$X^*=\mathcal L(X,\mathbb R)$$
is a Banach, with:
$$\Vert L\Vert _{X^*}=\Vert L\Vert_*=\sup_{\Vert x\Vert \leq 1}|Lx|$$
\paragraph{Example}
$$X=L^p(\Omega,\mathcal M,\mu)\quad 1\leq p\leq +\infty$$
Take:
\begin{itemize}
    \item $p'$ conjugate exponent: $\frac 1p+\frac 1{p'}=1$
    \item $u\in L^{p'}(\Omega, \mathcal M,\mu)$
\end{itemize}
Define:
$$L_u\in (L^p(\Omega))^*$$
$$L_uv=\int_\Omega uv\ \mathrm d\mu\quad \forall v\in L^p(\Omega)$$
Show that:
\begin{enumerate}
    \item $L_u$ is well defined
    \item $L_u$ is linear
    \item $L_u$ is continuous (bounded)
    \item $\Vert L_u\Vert _*=?$
\end{enumerate}
\textbf{Everything will follow from the Hölder's inequality} $$|L_uv|=\Big|\int_\Omega uv\ \mathrm d\mu \Big|\leq \Vert u\Vert_{p'} \cdot \Vert v\Vert_p $$
\begin{enumerate}
    \item $u\in L^{p'},\ v\in L^p\implies |L_uv|<+\infty\implies L_uv\in \mathbb R \quad \forall v$
    \item $L_u(\alpha_1v_1+\alpha_2v_2)=\int_\Omega u (\alpha_1v_1+\alpha_2v_2)\mathrm d\mu=\alpha_1 \int_\Omega uv_1\mathrm d\mu+\alpha_2\int_\Omega uv_2\mathrm d\mu=\alpha_1L_uv_1+\alpha_2L_uv_2$
    \item From Hölder's inequality $\implies |L_uv|\leq M\Vert v\Vert_p$, where $M=\Vert u\Vert _{p'} \implies \Vert L_u\Vert _*\leq \Vert u\Vert _{p'}$
    \item assume $1<p<+\infty$
    $$\Vert L_u\Vert_{(L^p)^*}=\sup_{v\neq 0} \frac{|L_uv|}{\Vert v\Vert _p}\begin{cases}
        \leq \sup_{v\neq 0} \frac{\Vert u\Vert _{p'}\Vert v\Vert _p}{\Vert v\Vert _p}=\Vert u\Vert _{p'}\\
        \geq \frac{|L_u\bar v|}{\Vert \bar v\Vert_p} \quad \forall \bar v\neq 0
    \end{cases}$$
    Choose $\bar v$ in such a way that $u\bar v=|u|^{p'}$
    $$\bar v = |u|^{\frac{p'}p}\cdot \mathrm{sign}(u)$$
    Then $\bar v\in L^p\impliedby u \in L^{p'}$\\
    and $$\Vert L_u\Vert_{(L^p)^*}\geq \frac{|L_u\bar v |}{\Vert \bar v\Vert_p}= \frac{\int_\Omega u\bar v\mathrm\ d\mu}{\Big (\int_\Omega |\bar v|^p\mathrm\  d\mu\Big )^{\frac 1p}}=\frac{\int |u|^{p'}\mathrm d\mu}{\Big (\int_\Omega |u|^{p'}\mathrm d\mu \Big )^{\frac 1p}}$$
    $$ = \frac{\Vert u\Vert_{p'}^{p'}}{\Vert u\Vert _{p'}^{\frac {p'}p}}=\Vert u\Vert _{p'}^{p'-\frac{p'}p}=\Vert u\Vert_{p'}$$
    (Note that $p'\Big (1-\frac 1p\Big )=p'\cdot \frac 1{p'}=1$)
    
\end{enumerate}

\textbf{Resuming}:
$$ 1<p<+\infty \implies \Vert L_u\Vert_{(L^p)^*}=\Vert u\Vert _{L^{p'}}$$
\paragraph{Question (and answer)}
Are all the elements of $(L^p)^*$ of type $L_u$, for some $u\in L^{p'}$? Yes, $1<p<+\infty$ (will be seen in the following)-
\paragraph{Remark}
The case $p=1,\ p'=+\infty,\ p=+\infty,\ p'=1$ are more delicate.\\In any case $L_u\in (L^p)^*$.
$$p=+\infty\implies \Vert L_u\Vert_{(L^\infty)^*}=\Vert u \Vert_{L^1}$$
$$p=1,\ X\text{ is }\sigma\text{-finite}\implies \Vert L_u\Vert_{(L^1)^*}=\Vert u\Vert_{L^\infty}$$
To provide the answers above, we need:
\begin{itemize}
    \item Hahn-Banach theorems
    \item Reflexive spaces
\end{itemize}
\subsection{Hahn-Banach Theorems and consequences}
\subsubsection{(thm) Hahn-Banach continuous extencion theorem}
Let $X$ be a normed space,
\begin{itemize}
    \item $Y\subset X$ subspace
    \item $L_0\in Y^*$
\end{itemize}
Then there exists $\tilde L_0\in X^*$ such that:
$$\begin{cases}
    \tilde L_0y=L_0y\quad \forall y\in Y\\
    \Vert \tilde L_0\Vert_{X^*}=\Vert L_0\Vert _{Y^*}
\end{cases}$$
\begin{proof}
    Omitted. Based on the axiom of choice.
\end{proof}
\paragraph{Corollary \#1}\ \\
$X$ normed space, $x_0\in X\setminus \{0\}$.\\
Then $\exists L\in X^* $ s.t. 
$$\begin{cases}
    \Vert L\Vert_{X^*}=1\\
    Lx_0=\Vert x_0\Vert _X
\end{cases}$$
\begin{proof}
    Take $Y=\mathrm{span}\{x_0\}=\{tx_0:t\in \mathbb R\}$. Since $Y$ is a one-dimensional subspace of $X$, the dual space $Y^*$ consists of all linear functionals on $Y$. 
    
    Define $L_0(tx_0)=t\Vert x_0\Vert_X$

    
\textbf{    is $L_0\in Y^*$?}    
\begin{enumerate}
        \item $L_0$ is a functional on $Y$
        \item $L_0$ is linear (check it!)
        \item $L_0$ is continuous (since in $\dim(Y)<+\infty$, all linear maps on $Y$ are continuous)
        \item The norm of $L_0$ in $Y^*$ is given by:
        $$\Vert L_0\Vert_Y=\sup_{\substack{y\in Y\\ y\neq 0}}\frac{|L_0y|}{\Vert y\Vert_Y}=\sup_{t \neq 0}\frac{|L_0(tx_0)|}{\Vert tx_0\Vert_X}$$
        By the definition of $L_0$, we have $L_0(tx_0)=t\Vert x_0\Vert_X$, and the norm $\Vert tx_0\Vert_X=|t|\Vert x_0\Vert_X$. Substituting into the expression:
        $$\sup_{t \neq 0}\frac{|L_0(tx_0)|}{\Vert tx_0\Vert_X}=\sup_{t\neq 0} \frac{|t|\Vert x_0\Vert}{|t|\Vert x_0\Vert}=1$$
        
    \end{enumerate}
    (HB)$\implies \tilde L_0x_0=L_0x_0=1\cdot \Vert x_0\Vert $\\
    Choose $L=\tilde L_0$
\end{proof}
\paragraph{Corollary \#2 (Bounded Linear Functionals separate points)}\ \\
$\forall x,y\in X$ normed space,
$$x\neq y\implies \exists\ L\in X^*:Lx\neq Ly$$
($Lx=Ly\quad \forall L\in X^*\implies x=y$)
\begin{proof}
    Take $x\neq y$ and apply Corollary \#1 to $x_0=x-y\neq 0$.\\
    $\exists L: Lx-Ly=L(x-y)=\Vert x-y\Vert \neq 0$ i.e. $Lx\neq Ly$
\end{proof}
\paragraph{Corollary \#3 (Bounded Linear Functionals separate closed subspaces and points)}
$X$ is normed, $Y\subsetneq X$ closed subspace and 
$x_0\in X\setminus Y$. Then $\exists L\in X^*$ s.t. $$\begin{cases}
    L_y=0 \quad \forall y \in Y\\ L_{x_0}\neq 0
\end{cases}$$
\paragraph{Remark ($\oplus$ notation)}
Consider $V,W\subset X$ subspaces, $V\cap W=\{ 0\}$. Then $V\oplus W=\{ v+w:v\in V, w\in W\}$
\begin{proof} (Corollary \#3)\\
    Take $Z=\spann\{x_0,Y\}=\spann\{ x_0\} \oplus Y = \{ z\in X:z=tx_0+y,\ t\in \mathbb R, \ y\in Y\}$\\
    Since $x_0\notin Y$ for every $z\in Z$, then $t,y$ are uniquely defined:
    $$t_1x_0+y_1=t_2x_0+y_2$$
    $$(t_1-t_2)x_0=y_2-y_1=0$$
    The first term belongs to $\spann\{x_0\}$, the second term belongs to $Y$.
    $$\implies t_1=t_2,\ y_1=y_2$$
    $$L_0:Z\to \mathbb R$$
    $$L_0( tx_0+y)= t\in \mathbb R$$
    We have to show that $L_0\in Z^*$
    \begin{enumerate}
        \item $L_0$ is linear
        \item $L_0$ is continuous (use the Closed Graph Theorem)
        $$\begin{cases}
            z_n\to \bar z\\ t_n\to \bar t
        \end{cases}\implies \bar z= \bar t x_0+\bar y$$
        
    \end{enumerate}
    $$ L(z_n)\to \bar t$$
    and $$\begin{cases}
        L_0x_0=L_0(1\cdot x_0+0)=1\\ L_0 y=L_0(0\cdot x_0+y)=0\quad \forall y\in Y
    \end{cases}$$
    and finally extend it to $L=\tilde L_0$ using Hahn-Banach.
\end{proof}
