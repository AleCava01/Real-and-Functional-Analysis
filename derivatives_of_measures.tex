\section{Derivatives of measures}
Consider $(X,\mathcal M,\mu)$ complete measure space.\\
We know that, given $\Phi :X\to [0,+\infty]$ measurable, the function
$$\nu_\Phi(E) \coloneqq \int_E \Phi \ \mathrm d\mu=\int_E \mathrm d\nu_\Phi$$
is a measure on $(X,\mathcal M)$.
\subsubsection{(def) Radon-Nikodin derivative (density for $\nu$ w.r.t. $\mu$)}
Consider $\mu, \nu$ measures on $(X,\mathcal M)$. If 
$$\exists\ \Phi \text{ s.t. } \nu(E)=\int_E \Phi\ \mathrm d\mu\quad \forall E\in \mathcal M$$
Then $\Phi$ is the Radon-Nikodym derivative of $\nu$ w.r.t. $\mu$:
$$\Phi = \frac{\mathrm d\nu}{\mathrm d\mu}$$
$\Phi:X\to[0,+\infty]$ acts as density for $\nu$ w.r.t. $\mu$, if fact, it expresses $\nu$ in terms of $\mu$ through the integral.
\subsubsection{(def) Absolute Continuous (AC) Measures}
Given $\mu, \nu$ measures on $(X,\mathcal M)$.\\
Then $\nu$ è assolutamente continua w.r.t. $\mu$ (notazione: $\nu << \mu$) se, 
$$\forall E \in \mathcal M,\ \mu(E)=0\implies \nu(E)=0$$
\subsubsection{(lemma) Absolute Continuity Lemma via Density Function 
 (*)}
$$\exists\ \Phi \text{ s.t. }\nu(E)=\int_E \Phi\ \mathrm d\mu\quad \forall E\in \mathcal M\implies \nu<<\mu$$
So the existance of the Radon-Nikodym derivative implies the absolute continuity of $\nu$ w.r.t. $\mu$.
\begin{proof}
 This holds because if $\nu(E)=\int_E\Phi\ \mathrm d\mu$ for all measurable sets $E$, then:
\begin{itemize}
    \item Whenever $\mu(E)=0$, the integral $\int_E \Phi \ \mathrm d\mu$ must also be zero because there is "no mass" in $E$ w.r.t. $\mu$. Consequently, $\nu(E)=0$ as well.
    \item This means that $\nu(E)=0$ whenever $\mu(E)=0$, which is precisely the definition of absolute continuity of $\nu$ w.r.t. $\mu$.
\end{itemize}   
\end{proof}


\subsubsection{(thm) Radon-Nikodym Theorem}
Consider:
\begin{itemize}
    \item $(X,\mathcal M)$ measurable space,
    \item $\mu, \nu$ measures,
    \item $\mu$ is $\sigma -$finite
\end{itemize}
Then:
$$\nu<<\mu\iff \exists\frac{\mathrm d \nu}{\mathrm d\mu}$$
\paragraph{Corollary} \ \\
$\nu$ measure on $(\mathbb R^n,\mathcal L(\mathbb R^n))$ and $\nu<<\lambda\implies$
$$\exists\ \Phi : \nu(E)=\int_E \Phi\ \mathrm d\lambda \quad \forall E\in \mathcal L(\mathbb R^n)$$
(Indeed, $\lambda $ is $\sigma-$finite)
