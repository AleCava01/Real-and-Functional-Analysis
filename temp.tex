\section{Temp}
\subsubsection{(thm) (*)}
Dati
$$\begin{cases}\forall E \in \mathcal L([a,b])\\\lambda(E)<\delta\end{cases} \implies \int_E |f| d\lambda <\varepsilon$$

\subsubsection{(examples) AC$\not\implies$ Lipschitz continuous}
\textbf{Esempio 1}
Consider $f(x)=\sqrt x $ in $[0,1]$
$$\sqrt x = \int _0^x \frac 1{2\sqrt t }\mathrm dt\quad x\in [0,1]$$
e $g\in AC([0,1])$. (Indeed, $g'\in L^1(0,1)$
but $\g\notin $Lipschitz continuous\\
\subsubsection{(example) UC$\not\implies$ AC}
Considera $$g(x)=\begin{cases}x \sin{1/x}\quad 0<x\leq 1\\0\quad x=0\end{cases}$$
è continua in $[0,1]\implies g\in UC([0,1])$
But it is not AC.\\
Indeed,
$$ g'(x)=\sin (\frac 1x)-\frac 1x \cos(\frac 1x)\quad 0<x\leq 1$$
\begin{itemize}
    \item $\sin (\frac 1x)\in L^1([0,1])$
    \item $-\frac 1x \cos(\frac 1x) \notin L^1([0,1])$
\end{itemize}

\subsection{AC functions and weak derivatives}
Lavoriamo in $X=[a,b]\subset \mathbb R$ (diventa molto diverso in $\mathbb R^n$).
\subsubsection{(prop) intergration by parts in AC}
Take $u\in [a,b]\to \mathbb R$\\
$u\in AC([a,b])\iff$
\begin{itemize}
    \item $u \in C([a,b])$
    \item u è differenziabile a.e.
    \item $u'\in L^1([a,b])$
    \item $$\int_a^b u'\varphi \mathrm d x = -\int_a^b u\varphi'\mathrm d x\quad \forall \varphi \in C_0^\infty([a,b])$$
\end{itemize}
\subsubsection{(recall) Compactly supported ($\varphi \in C_0^\infty([a,b])$)}
\begin{itemize}
    \item $\varphi \in C^\infty$
    \item $\exists [c,d]\subset (a,b)$ t.c. $\varphi=0 $ in $ (a,b)\setminus [c,d]$
\end{itemize}
\subsubsection{(def)}
Let $u\in L^1(a,b)$. Diciamo che $u \in W^{1,1}(a,b) \iff \exists w\in L^1(a,b) $ t.c. $\int_a^b u\varphi '\mathrm dx=-\int_a^b w\varphi \mathrm d x \quad \forall \varphi \in C_0^\infty $

Such a $w$ is called weak derivative of $u$ in $(a,b)$, denoted $u'$

\subsubsection{(rmk)}
Both $u$ and $w=u'$ are equivalence classes
\subsubsection{(rmk)}
If such a $w$ exists, it is unique.
$$\int_a^b u \varphi ' = -\int_a^b w_1\varphi $$
$$\int_a^b u \varphi ' = -\int_a^b w_2\varphi $$
$\forall \varphi \in C_0^\infty ([a,b])$
$$\implies\int_a^b(w_1-w_2)\varphi =0\quad \forall \varphi \in C_0^\infty$$
$$\implies w_1-w_2=0\quad a.e. \ in \ [a,b]$$
\subsubsection{(rmk)}
In principle, the pointwise and weak derivatives are different objects, and the notation $u'$ may be misleading.\\
But we know that if we take absolute continuous functions, they coincide.
\subsubsection{(rmk)}
The definition of weak derivative can be extended in the senses of measures, of distributions.

Take $$H(x)=\begin{cases} 0\quad x<0\\ 1\quad x>0\end{cases}$$
$$-\int_{-1}^1 H(x)\varphi '(x)\mathrm d x = -\int_0^1 \varphi'(x)\mathrm d x = -\varphi(1)+\varphi(0)$$
$$\varphi(1)=0$$

$$\varphi(0)=\int_[-1,1] \varphi \mathrm d\delta_0$$
This suggests that $H^1$ è pari a 0 almost everywhere (pointwise), ma è pari a $\delta_0$ (dirac) weakly
\subsubsection{(thm)(*)}
$$u\in  AC([a,b])\iff u \in W^{1,1}([a,b])$$

Devi farti la dimostrazione.
